\documentclass{beamer}

\usepackage{amsmath}

\usetheme{AnnArbor}
\usecolortheme{crane}
\usefonttheme[onlymath]{serif}

\title{Deep Learning - Foundations and Concepts}
\subtitle{Chapter 10. Convolutional Networks}
\author{nonlineark@github}
\date{\today}

\begin{document}

\begin{frame}
    \titlepage
\end{frame}

\begin{frame}
    \frametitle{Outline}
    \tableofcontents
\end{frame}

\section{Computer Vision}

\begin{frame}
    \frametitle{Computer vision}
    \begin{itemize}
        \item Computer vision was one of the first fields to be transformed by the deep learning revolution, predominantly thanks to the CNN architecture.
        \item Recently alternative architectures based on transformers have become competitive with convolutional networks in some applications.
        \item Some applications for machine learning in computer vision: Classification, detection, segmentation, caption generation, synthesis, inpainting, style transfer, super-resolution, depth prediction, scene reconstruction.
    \end{itemize}
\end{frame}

\begin{frame}
    \frametitle{Image data}
    \begin{itemize}
        \item The structure of an image:
        \begin{itemize}
            \item An image comprises a rectangular array of pixels.
            \item Each pixel has either a grey-scale intensity or a triplet of red, green and blue channels each with its own intensity value.
        \end{itemize}
        \item Challenges of applying neural networks to image data:
        \begin{itemize}
            \item Images generally have a high dimensionality.
            \item Image data is highly structured.
        \end{itemize}
        \item Local correlations can be used to encode strong inductive biases into a neural network, leading to models with far fewer parameters and with much better generalization accuracy.
    \end{itemize}
\end{frame}

\end{document}